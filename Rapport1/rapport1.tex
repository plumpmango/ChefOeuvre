\documentclass[a4paper,12pt]{report}

\usepackage[utf8]{inputenc}
\usepackage[frenchb]{babel}
\usepackage[utf8]{inputenc}
\usepackage[T1]{fontenc}
\usepackage{geometry}
\usepackage{fancyhdr}
\usepackage{pdfpages}
\usepackage{hyperref}
\usepackage{pgf,tikz}
\usepackage{amsmath}
\usepackage{algorithm}
\usepackage{algorithmic}
\usepackage{epsfig}
\usetikzlibrary{arrows}

%Initialisation des entêtes et pieds de page
\pagestyle{fancy}
\renewcommand{\headrulewidth}{1pt}
\fancyhead[C]{}
\fancyhead[L]{IIG}
\fancyhead[R]{Rapport TP}
\renewcommand\footrulewidth{1pt}
\fancyfoot[C]{M2 IGAI - Université Paul Sabatier\\
\textbf{\thepage}}

\makeatletter
\renewcommand{\@chapapp}{}
\makeatother

%Faire dépendre les sections des partie pour avoir une numérotation à partir de 1
\makeatletter\@addtoreset{section}{part}\makeatother
\renewcommand{\thesection}{\arabic{section}}

\title{\LARGE \textbf{RAPPORT\\ Chef d'O{e}uvre \\ Méthodes et Algorithmes }}
\author{\textsc{Courdy-Bahsoun Clémence - Deker Sylvain}\\
\textsc{Kottath Sandeep - Moussa Nahor - Yazi Ryma}\\\\
M2 IGAI  \\
}
\date{23.11.2019}


\begin{document}
\maketitle
\newpage


\section{Introduction}

\paragraph{}
Le projet proposé par le client, Mr Alain Crouzil, est le développement d'un outil de ré-identification de véhicule dans des enregistrements. Plus précisément il s'agit de retrouver dans un ensemble de vidéo le véhicule correspondant à celui recherché. Ainsi l'objectif est de faire un éditeur permettant de retrouver un véhicule dont l'image est donnée en paramètre dans des vidéos de surveilances fournit par une base de donnée. Pour ce faire il est imposée d'utiliser une approche de résolution par apprentissage profond en utilisant des réseaux de neurone existant comme le propose les détecteurs d'objets de type YOLO et SSD. Il sera donc nécessaire d'entrainer le réseau avec les données qui nous intéresse et pour cela une base d'apprentissage riche en libre accès est à notre disposition. Après quoi en mesurant la différence entre les résultats obtenus en avant dernière couche en sortie du réseau pour l'image de référence (véhicule recherché) et les résultats obtenu permettra de trier ces derniers en fonction de leur pertinence, et ainsi d'afficher le ou les résultats les plus probants.
% \paragraph{}
% Plan de développement 
%     -> Découpage pour la réalisation du projet : étapes clés
%     -> planning prévisionnel
%     -> Organisation (communication, gestion des versions, des problèmes et des tests)
    
    \paragraph{}
    Méthodes et algorithmes 
    
    

\section{Méthodes et algorithmes}



\section{Conclusion}




\end{document}

